% コンパイル方法: lualatex filename.tex
\RequirePackage{plautopatch}

\documentclass[a4paper, 10pt]{ltjsarticle}


% マージン設定
\usepackage[top=20mm, bottom=20mm, left=20mm, right=20mm]{geometry}

% LuaLaTeX用日本語対応パッケージ
\usepackage{luatexja}
\usepackage{luatexja-fontspec}

% 必要なパッケージ
\usepackage{fontspec}
\usepackage{titlesec}
\usepackage{graphicx}
\usepackage{amsmath}
\usepackage{amssymb}
\usepackage{hyperref}
\usepackage[english, japanese]{babel}
\usepackage{multicol} % 二段組用パッケージ
\usepackage{indentfirst}
\usepackage{tikz} % カスタム点線用
\usepackage{authblk} % 著者・所属パッケージ
\usepackage{here}
\usepackage{caption}
\usepackage{tabularx}
\usepackage{subcaption}

% \setmainfont[Ligatures=TeX]{Times New Roman}
% \setmainjfont[BoldFont=MS Gothic]{MS Mincho}

\renewcommand{\baselinestretch}{0.95}

% セクション見出しのカスタマイズ
\titleformat{\section}
  {\fontsize{10pt}{10pt}}
  {\thesection.}
  {1em}{}

\titleformat{\subsection}
  {\fontsize{10pt}{10pt}}
  {\thesubsection}
  {1em}{}

\titleformat{\subsubsection}
  {\fontsize{10pt}{10pt}}
  {\thesubsubsection}
  {1em}{}

  \setlength{\parindent}{1em}

\captionsetup[table]{skip=0pt}
\captionsetup[subfigure]{font=normalsize}
\setlength{\belowcaptionskip}{0.5em} % キャプション下の余白を設定


\titlespacing*{\section}{0em}{1em}{0em}
\titlespacing*{\subsection}{0em}{1em}{0em}

\pagestyle{empty}


\begin{document}

\setlength{\abovedisplayskip}{-0.5em}
\setlength{\belowdisplayskip}{0.5em}
% \setlength{\columnsep}{7.5mm}

\twocolumn[
    \begin{center}
        {\vspace{-1em}}

        {\fontsize{15pt}{15pt}\selectfont{クロスレイヤシミュレータにおける無線LAN評価モデルの検討}}

        {\vspace{1.3em}}

        {\fontsize{13pt}{13pt}\selectfont{A Study of a Wireless LAN Evaluation Model in a Cross-Layer Simulator}}
    \end{center}

    \vspace{0.1em}

    \begin{flushright}
      {\fontsize{11pt}{11pt}\selectfont{T5-16 \, 下沢亮太郎}}
      \\
      {\fontsize{11pt}{11pt}\selectfont{指導教員 \, 設樂勇}}
    \end{flushright}

    \vspace{1em}

    \thispagestyle{empty}
]

\section{はじめに}
% 近年,無線通信端末利用者の急増に伴い様々な場所で無線通信システムが利用されており,今後も利用の増加と発展が見込まれている.近年の無線通信技術の進歩に伴い,システムが高機能化・複雑化しており,従来のようにレイヤごとに独立した性能評価を行う手法では通信全体の実用的な評価を十分に行うことが困難になりつつある.そのため,通信全体をクロスレイヤで評価できる計算機シミュレータの開発が求められている.本研究ではクロスレイヤシミュレータにおけるMACレイヤの挙動をシミュレートする機能の開発を行い,その有効性を評価することを目指した.
% 従来のように,各レイヤを独立して評価する手法では,通信全体を俯瞰した実用的な性能評価が難しくなりつつある.そのため,物理層から上位層までを跨いで統合的に評価できるクロスレイヤシミュレータの開発が求められている.
% 本研究では,クロスレイヤシミュレータの一部としてMAC(Medium Access Control)レイヤの挙動をシミュレートする機能を開発し,その有効性を評価することを目的とする.
% しかし,各レイヤごとに性能のボトルネックが存在するため,他レイヤとの統合した評価が必要不可欠である.
% 本研究では,IEEE 802.11規格に基づくCSMA/CA(Carrier Sense Multiple Access with Collision Avoidance)方式を用いたMAC(Medium Access Control)レイヤの挙動を再現したモデルを構築し,クロスレイヤシミュレータの一部として動作を再現するシミュレータを開発した.その上で,開発したシミュレータの動作が理論値と一致することを評価し,その有効性を検証することを目指す.


近年,無線通信端末の利用者が急増し,多様な場所で無線通信システムが活用されており今後も利用の拡大と機能の高度化が見込まれる.一方で,無線通信技術の進歩に伴いシステム自体はますます高機能化・複雑化している.
しかし,研究開発の現場では各レイヤごとに検討が行われており,単一レイヤのみの評価では通信システム全体の性能を十分に把握することができない.
本研究では,無線通信全体の品質を総合的に評価するために,実環境の電波伝搬特性を考慮した物理層とMAC(Medium Access Control)層が連携したシミュレータの開発の一環として,IEEE 802.11規格に基づくCSMA/CA(Carrier Sense Multiple Access with Collision Avoidance)方式を用いたMAC層の動作に則った無線LAN(Local Area Network)モデルを開発し,その精度を評価することを目指す.


\section{無線LAN通信モデル}


\subsection{CSMA/CA方式}

% CSMA/CAでは,送信したいフレームが発生した際にCarrier Sense(CS)を行い,チャネルが空いているとき(Idle)バックオフ時間という各端末がランダムなスロット数を生成し,それに従った待ち時間を待った後,再度キャリアセンスを行いチャネルがIdleであることを確認してからフレームを送信し,Busyだった場合はフレームを送信できるまでバックオフ時間を持ち越す.複数の端末が同じスロット数を生成した場合には送信タイミングが重なり,衝突が発生するため再送処理が必要となる.

IEEE 802.11規格では,CSMA/CAと呼ばれるアクセス制御方式を採用している.図\ref{CSMA/CA}にCSMA/CAの概要を示す.
CSMA/CAでは,送信したいフレームが発生した際,まずキャリアセンス(CS)を行い,チャネルが空いている(Idle)かどうかを確認する.
チャネルがIdleの場合は,各端末がバックオフ時間としてランダムなスロット数を生成し,その時間だけ待った後に再度キャリアセンスを行い,チャネルがIdleのであればフレームを送信し,使用中(Busy)だった場合はフレームを送信できるまでバックオフ時間を持ち越す.複数の端末が同じスロット数となった場合には送信タイミングが同時になり,衝突が発生するため再送制御が必要となる.

無線LANシステムでは2進数バックオフ方式を採用している.バックオフ制御に用いるContention Window(CW)は,最小値を15とし,最大値の上限を1023スロットとして衝突回数の増加に従いCWサイズは.再送回数を$n$とすると,CWの最大値$\mathrm{CW}_{\max}$とスロット数$s$は


\begin{align}
  \mathrm{CW}_{\max} &= 2^{4 + n} - 1
\end{align}

\begin{align}
  s &= \mathrm{randint}(1, \, \min(c_{\max}, \, 1023))
  \label{slot}
\end{align}

で求められる.

衝突が発生するたびにCWの最大値は2倍に増加するため,再送回数が増えるほどバックオフ時間が長くなることで他端末と同じCWサイズを生成することがなくなり,衝突の発生を抑制することができる.一方で,CWの増加がオーバーヘッドを引き起こし,スループットの低下につながる.


% 本シミュレータでは,端末クラスにスロット生成メソッドを実装し,インスタンスごとにスロット数と再送回数$n$を保持することで,各端末が送信を試みる際の待機時間を動的に設定する処理を実装した.

本シミュレータでは,端末単位でスロット数と再送回数$n$を保持することで,各端末が送信を試みる際の待機時間を動的に設定する処理を実装した.


\begin{figure}[htbp]
  \centering

  \begin{subfigure}{\columnwidth}
    \centering
    \includegraphics[width=1\columnwidth]{./assets/csma-ca-s.png}
    \caption{CSMA/CA成功例}
    \label{1a}
  \end{subfigure}


  \begin{subfigure}{\columnwidth}
    \centering
    \includegraphics[width=1\columnwidth]{./assets/csma-ca-f.png}
    \caption{CSMA/CA失敗例}
    \label{1b}
  \end{subfigure}


  \caption{CSMA/CAの概要}
  \label{CSMA/CA}
\end{figure}

\vspace{-3em}

\subsubsection{IFSによる優先制御}

フレーム間にはIFS(Inter Frame Space)と呼ばれる待機時間が設定されている.IFSの長さは6種類存在し,代表的なものにDIFS(Distributed Inter Frame Space)とSIFS(Short Inter Frame Space)がある.これらは,フレームの優先順位に基づいてどのIFSを選択するかが決定される.

DIFSはデータフレーム送信時に適用されるIFSであり,端末は送信開始前にDIFS時間($34\mathrm{\mu s}$)待機してからデータフレームを送信する.
一方,ACK(ACKnowledgment)フレームのように失敗すると再送制御が必要となる優先度の高い制御フレームは,DIFS時間待つと他端末のデータフレームと競合する可能性があるため,より短いSIFS時間($16\mathrm{\mu s}$)を用いることで優先的に送信するように設定されている.

% 一方SIFS時間($16\mathrm{\mu s}$)はDIFSよりも短い待機時間として設定されており,データフレームの応答確認に用いられ,失敗すると再送処理が必要となるACK(ACKnowledgment)フレームなどの優先順位の高い制御フレームの送信間隔として用いられる.

% ACK(ACKnowledgment)フレームなど,優先度の高い制御フレームの送信間隔として用いられるIFSであり,受信が成功したことを送信元へ通知し再送処理を減らすために使われる.

% SIFSは,ACK(ACKnowledgment)フレームなど,優先度の高い制御フレームの送信間隔として用いられる最短のIFSである.一方,DIFSはデータフレーム送信時に適用されるIFSであり,SIFSよりも長い時間に設定されている.CSMA/CAでは,端末がチャネルを利用する前にDIFSだけ待機する必要があり,チャネルが空いていればデータフレームを送信する.


\subsection{フレーム構成モデル化}
% UDPレベルでの無線LAN通信をシミュレートするために,簡易的にフレームを構成するモデルを作成した.図\ref{packet}に,モデル化されたフレームの構成図を示す.

本研究では,UDP(User Datagram Protocol)レベルの無線LAN通信を再現するためにフレーム構成をモデル化した.図\ref{packet}にモデル化したフレームの構成図を示す.


無線LAN通信では,データ送受信時にPLCP(Physical
Layer Convergence Protocol)プリアンブルやMACヘッダ,FCS(Frame Check Sequence)などの制御情報のオーバーヘッドに加え,ACKフレームの送信やCSMA/CA特有のDIFS・SIFSなどのフレーム間隔,バックオフ動作も必要となる.

このため,物理層のみを考慮したシミュレーションと比べ,実環境でのスループットは低下することから,クロスレイヤでの検討が必要である.

\begin{figure}[htbp]
  \centering
  \includegraphics[width=1\columnwidth]{./assets/packet.png}
  \caption{フレーム構成図}
  \label{packet}
\end{figure}

\vspace{-2em}

% PLCPプリアンブル,MACヘッダ,FCSやACKフレーム,DIFS,SIFSなどのオーバーヘッドがあるため,物理層のシミュレーションよりも実際のスループットは小さくなる.

% 実際の無線LAN通信では,データフレームを送受信する際にPLCPプリアンブルやMACヘッダ,FCSといった制御情報を付加する必要があるほか,受信側からのACKフレーム送信も必須となる.また,CSMA/CA方式特有のDIFSやSIFSなどのインターフレームスペースを設ける必要があり,フレーム衝突を回避するためのバックオフ動作も発生する.これらの各種オーバーヘッドと制御処理が加わる結果,純粋な物理層の伝送速度のみを用いたシミュレーション結果と比較すると,実環境での実効スループットは必然的に低下する.



% \section{提案手法}
% \subsection{User class}

\vspace{-0.5em}

\section{実装とシミュレーション設定}

本研究では,CSMA/CA方式を用いた無線LANを再現するために,Pythonを用いてシミュレータを作成した.シミュレータには,各端末を管理する端末クラスを導入し,端末ごとのCWや再送回数の管理,バックオフ時間を決定するためのスロット数の管理,再送処理などの機能を実装している.

また,シミュレータ本体は標準ライブラリである\texttt{random}のみに依存するように設計した.これにより,バージョン差異による影響を受けにくい後方互換性のあるシミュレータを実現した.


% \subsection{\texttt{User}クラスの設計}
% 本研究のシミュレータ実装では,各端末を\texttt{User}クラスとして定義し,端末ごとのCWや再送回数などを管理している.表\ref{tab:user-class}に主なメンバ変数と役割をまとめる.

% \begin{table}[H]
%   \centering
%   \caption{Userクラスのメンバ変数とメソッドの一覧}
%   \label{tab:user-class}
%   \begin{tabularx}{\columnwidth}{lX}
%     \hline
%     % \textbf{名称} & \textbf{説明} \\
%     名称 & 説明 \\
%     \hline
%     \multicolumn{2}{l}{メンバ変数} \\
%     \hline
%     \texttt{id} & 端末を識別するためのID\\
%     \texttt{num\_re\_trans} & 再送回数\\
%     \texttt{slots} & スロット数\\
%     \texttt{num\_transmitted} & 送信成功回数\\
%     \texttt{data\_transmitted} & 送信したデータ量 \, [bit]\\
%     \hline
%     \multicolumn{2}{l}{主なメソッド} \\
%     \hline
%     \texttt{calc\_slots()} &(\ref{slot})式に従い\texttt{slots}を決定\\
%     \texttt{re\_transmit()} & 再送処理\\
%     \texttt{reset\_slots()} & 新たに\texttt{slots}を割り当てる\\
%     \hline
%   \end{tabularx}
% \end{table}

\vspace{-0.5em}

\subsection{シミュレーション条件}
表\ref{tab:sim-param}に,本研究で用いシミュレーション条件を示す.
モードを選択することでそれぞれの無線LAN規格(IEEE 802.11a/b/g)に対応できるように設計した.

% 無線LAN規格(IEEE 802.11a/b/g)によりスロット時間やDIFS/SIFSなどのフレーム間隔が異なるため,対応するモードを選択することができる.

\vspace{-1em}


\begin{table}[htbp]
  \centering
  \caption{シミュレーション条件の例}
  \label{tab:sim-param}
  \begin{tabular}{c|@{\hspace{1.8em}}l}
    \hline
    パラメータ & 値・例 \\
    \hline
    シミュレーション時間 & 60 \, \,$\mathrm{s}$\, \\
    スロット時間 (802.11a) & \, 9 \, \,$\mathrm{\mu s}$\, \\
    DIFS (802.11a) & 34 \, \,$\mathrm{\mu s}$\, \\
    SIFS (802.11a) & 16 \, \,$\mathrm{\mu s}$\, \\
    伝送レート & 24 \, \,Mbps\, \\
    端末数 & 80 \, \,台\, \\
    \hline
  \end{tabular}
\end{table}

\vspace{-2em}

\section{評価}
横軸を端末数,縦軸をスループットとしたシミュレーション結果と理論値を図\ref{fig:simulation-result}に示す.



理論値との差が一番大きい端末数が80台の場合でも+2.75\%程度の誤差に収まっていることが確認できる.また,端末数が増加するにつれて理論値との差が徐々に拡大することに対しては,参考とした文献\cite{paper}とのIP(Internet Protocol)レベルとUDPレベルのプロトコル上の違いからくるペイロード長の差などモデル化方法の違いが影響していると考えられる.
以上の結果から,本研究で構築したCSMA/CAベースの無線LANモデルは,理論値に対して概ね一致し,最大でも誤差がおよそ3\%にとどまることが示された.

\begin{figure}[htbp]
  \centering
  \includegraphics[width=1\columnwidth]{./assets/g3.png}
  \caption{シミュレーション結果}
  \label{fig:simulation-result}
\end{figure}

\vspace{-2em}

\section{まとめ}
% 本研究では,クロスレイヤシミュレータにおける無線LANシミュレータを開発し,CSMA/CAを中心とした基本動作のモデル化とそのシミュレータの精度を検証した.
% 本研究では,クロスレイヤシミュレータにおける無線LANシミュレータを開発し,CSMA/CAの基本動作のモデル化とそのシミュレータの精度を検証した.

本研究では,クロスレイヤシミュレータにおけるCSMA/CAを中心とした無線LANシステムのモデル化とそのシミュレータを開発し,精度を検証した.

今後の課題としては、連続送信ではなくポアソン分布などに従った送信間隔を導入し、より実際の通信頻度に近い状況を再現することが挙げられる。
さらに、端末ごとに伝送速度を変えられるようにすることや、各端末やアクセスポイントの位置情報を踏まえて受信時のSNR(Signal-Noise Ratio)を考慮し、衝突時でもフレームの複合が可能となるキャプチャ効果を導入することで、より実環境に近い通信環境を再現することが挙げられる。

% 今後の課題としては,連続送信ではなくポアソン分布などに従った送信間隔を導入することで実際の通信頻度に近い状況を再現することや,端末ごとに伝送速度を変えられるようにすることが挙げられる。
% また,各端末やアクセスポイントの位置情報を考慮し,受信時のSNR(Signal-Noise Ratio)によって衝突時でもフレームの複合が可能になるキャプチャ効果も考慮できるようにすることでより実際の通信環境に近づけることも求められる.


% 衝突が起きても受信電力が干渉電力より十分に大きい場合,フレームの複合に成功するキャプチャ効果を各端末ごとに実装することが挙げられる.

% 端末ごとに伝送速度を変える
% 各端末やアクセスポイントの位置情報を考慮し,

% 端末間の距離による自由空間伝搬損失などの要素を考慮し,より実環境に近い評価を行う必要がある.

% 本研究では,クロスレイヤシミュレータの一部である無線LANシミュレータを開発し,CSMA/CAを中心とした動作のモデル化と検証を行った.

% また,現在の方法では通信が連続して行われているが,ポアソン分布に従った時間だけ離すことで実際の通信頻度に近い状況を再現することや,各ステーションに距離の概念を持たせて自由空間伝搬による減衰を考慮することが今後の課題である.

\vspace{-0.5em}

\begin{thebibliography}{9}
  \bibitem{midori}守倉正博, 久保田周治, 『インプレス標準教科書シリーズ 改訂三版802.11 高速無線LAN教科書』, 株式会社インプレスコミュニケーションズ, 2016年
  \bibitem{paper}Y. Morino, T. Hiraguri, H. Yoshino, K. Nishimori, T. Matsuda, ``A Novel Collision Avoidance Scheme Using Optimized Contention Window in Dense Wireless LAN Environments*'' \, \textit{IEICE TRANS. COMMUN.}, VOL.E99-B, NO.11 NOVEMBER 2016
  \bibitem{11std}IEEE 802.11 Standard for Local and Metropolitan Area
  Networks, “Part 11: Wireless LAN Medium Access Control (MAC) and Physical Layer (PHY) Specifications,”  IEEE Std. 802.11, Mar. 2012.
  % \bibitem{book1}西森健太郎,平栗健史,『MIMOからMassive MIMOを用いた伝送技術とクロスレイヤ評価手法』, コロナ社, 2017年.
  % \bibitem{book2}設樂勇, 平栗健史, 谷口諒太郎, 西森健太郎, 『レイトレースを用いた3次元クロスレイヤシミュレータの開発』, 社団法人 電子情報通信学会 信学技報
\end{thebibliography}



\end{document}
